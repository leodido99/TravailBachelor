\chapter{Concept}

L’idée du projet est de développer un système de suivi en temps réel de la compétition qui se base sur les informations transmises par un capteur porté par les sportifs. Cela permettrait de pouvoir à tout moment afficher la position actuelle des sportifs le long du parcours ainsi que d’autres informations intéressantes comme la vitesse moyenne, le rythme cardiaque et le nom du coureur par exemple. Le but étant, à l’aide d’une application mobile, de permettre aux spectateurs de pouvoir suivre le déroulement de l’événement plus aisément et de manière globale et ainsi d’essayer de rendre la compétition plus vivante pour lui.

Afin de pouvoir exploiter les données enregistrées par les capteurs, il faut les récupérer et les centraliser sur un serveur. Pour se faire une des solutions les plus communes serait d’utiliser le réseau de téléphonie mobile, ceci comporte un désavantage important qui est le besoin d’acquérir des cartes SIM pour chaque capteur pour pouvoir leur permettre de communiquer sur le réseau. D’autres part, suivant l’endroit où la course se déroule la couverture du réseau peut être faible voire inexistante. 

Pour pallier à ses défauts ce projet vise l’utilisation du protocole LoRa qui ne nécessite pas de surcoût pour son utilisation et dont les modules de transmission et réception ont un faible coût. Ensuite un système de borne posé à certains endroits le long du parcours, des passerelles, est utilisé pour garantir une couverture du réseau optimal. Ce système permet la flexibilité de rajouter ou au contraire d’enlever certaines passerelles en fonction de la couverture du réseau souhaitée. Grâce à ses passerelles le système est capable de récupérer les données transmises par les capteurs et de les stocker dans une base de données qui peut ensuite être exploité par une application.

Une autre plus-value de ce système par rapport aux solutions existantes actuellement est qu’il vise à proposer une mise à jour en temps réel des informations, c’est-à-dire que les données produites par les capteurs sont disponibles pratiquement instantanément aux utilisateurs au travers de l’application mobile associée, cela permet de visualiser l’évolution de la course avec plus de précision et d’interactivité. La majorité des solutions existantes actuellement enregistrent des données durant la course mais il faut attendre la fin de l’activité pour pouvoir les télécharger sur un site web afin de visualiser le parcours et les statistiques.