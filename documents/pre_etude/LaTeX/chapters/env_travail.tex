\chapter{Environnement de travail}

Le tableau~\ref{tab:env_travail} contient la liste des éléments faisant parti de l’environnement de travail. C’est-à-dire les différents outils et logiciels nécessaire au développement du travail de Bachelor.

\begin{table}[htb]
\caption[Environnement de travail]{Liste des outils utilisés durant le travail de Bachelor}
\label{tab:env_travail}
\centering
\begin{tabular}{lp{8cm}}
\toprule
Nom & Description \\
\midrule
Eclipse &	Environnement de développement open source pour C et C++\\
Zephyr ou mBed & Système d’exploitation temps réel utilisé par le capteur. Il dépendra du choix final du capteur.\\
GNU Compiler Collection (GCC) & Compilateur C et C++\\
Cross Compilateur ARM & Compilateur C et C++ pour la compilation du code source exécuté sur le capteur\\
Make & Exécute les fichiers makefile utilisés pour compiler le code source\\
Android Studio & Environnement de développement pour la plateforme Android\\
pgAdmin & Outils de développement pour les bases de données PostgreSQL\\
git	 & Outils de configuration\\
Semtech LoRa Packet Forwarder & Programme qui s’exécute sur la passerelle et transfère les paquets LoRa reçu au serveur de paquet. \\
\bottomrule 
\end{tabular}
\end{table}