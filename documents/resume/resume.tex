\LARGE{Résumé}

\begin{normalsize}

\todo{Faire signer}

Les compétitions sportives, de course à pied ou de cyclisme par exemple, sont des événements dont le déroulement n'est pas toujours facile à suivre pour les spectateurs qui sont sur place. Une fois le départ de la course donné, on peut vite perdre de vue les concurrents. Il faut donc se déplacer pour pouvoir suivre son déroulement et même dans ce cas, vu que le nombre de sportifs est souvent élevé, on peut difficilement se faire une idée globale sur la position de chacun d'eux ou du classement à un instant précis.

Ce projet propose la réalisation d’un système permettant le suivi en temps réel de compétitions sportives au travers d’une application mobile Android afin d'impliquer d'avantage les spectateurs. Grâce à une carte sur laquelle il est possible de voir le parcours de la course en entier, l'application tiendra à jour la position GPS de chaque coureur en temps réel au travers de données transmises par un capteur porté par les sportifs. L'interface propose également d'autres informations: le nom du sportif, son pays d'origine, son numéro de dossard, la distance parcourue, la vitesse moyenne, son rythme cardiaque et sa cadence (nombre de pas par unité de temps). Enfin, l'application permet de choisir ses coureurs favoris afin d'en faciliter le suivi ainsi que l'administration du système telles que la création des courses ou l'inscription des coureurs par exemple.

Pour que le système puisse remplir sa tâche, les éléments suivant ont été développés.

\begin{itemize}
\item Un capteur qui est porté par les athlètes et qui permet l'acquisition des données
\item Une passerelle qui est placée le long du parcours et qui centralise les données produites par les capteurs dans une base de données
\item Une application mobile qui permet la visualisation des données produites
\end{itemize}

La communication entre le capteur et la passerelle est assurée grâce à la technologie sans-fil LoRa, un protocole qui est prévu pour être utilisé par de petits systèmes avec peu de ressources sur des distances de plusieurs kilomètres, et qui ne nécessite pas de surcoût pour son utilisation contrairement aux communications mobiles GSM par exemple, car il utilise les bandes de fréquences libres ISM.

Le capteur dispose d'un micro-contrôleur ATSAMD21 avec un cœur ARM Cortex-M0+ sur lequel s'exécute le firmware développé pendant le travail de Bachelor. Il est écrit en C et utilise le système d'exploitation temps réel Zephyr qui propose tous les mécanismes et services nécessaires pour le développement d'une application embarquée temps réel.

La passerelle est construite à partir d'un Raspberry Pi 3 model B+ exécutant le système d'exploitation Linux ainsi que d'une carte de gestion de la couche radio LoRa. Une application serveur, codée en C++, se charge de récupérer les paquets reçus par l'interface LoRa et d'enregistrer les données décodées dans la base de données.

L'application mobile est développée en Java sur Android Studio et utilise le framework "Maps SDK for Android" de Google afin de gérer la carte et les éléments affichés dessus.

\end{normalsize}