
\chapter{Développement du projet}\label{ch:developpement}

Durant la pré-étude, j'ai pris la décision de découper le développement du projet en trois phases distinctes afin de pouvoir gérer le travail au mieux et pouvoir réagir rapidement en cas d'imprévus ou de problèmes. Pour chaque phase, des objectifs ont été établis afin de pouvoir, à son terme, en valider l'exécution et ainsi passer à la phase suivante.

Les sections suivantes décrivent les trois phases de développement du projet.

\section{Phase \#1}

\begin{table}[htb]
\caption{Dates clefs phase \#1}
\label{tab:detail_phase_1}
\centering
\begin{tabular}{ l l l }
\toprule
Début & Validation \\
\midrule
9 Juillet 2018 & 13 Juillet 2018  \\
\bottomrule 
\end{tabular}
\end{table}

La première phase de développement qui a commencé dès le début du projet, c'est à dire le 9 Juillet 2018, avait pour objectif la validation de la transmission des données avec la technologie LoRa et la solution matériel choisie pour le capteur et la passerelle. Durant cette phase, seuls le capteur et la passerelle ont été testés. 

Pour ce faire, un programme simple de test a été écrit pour le capteur en se basant sur le framework Arduino, facilitant et accélérant grandement le développement car déjà tous les modules et drivers nécessaires étaient disponibles, le but étant de permettre de faire des tests de transmissions de paquet rapidement et efficacement. En ce qui concerne la passerelle, le début du serveur d'application a été développé qui permet la récupération des données transmises par le capteur. Un gestionnaire spécial pour ce test a été développé qui permet de vérifier si des paquets ont été perdus lors de leur transmission afin de jauger de la qualité de la transmission LoRa.

Au terme de ce développement un test a été effectué en extérieur afin de s'assurer que le matériel était capable d'envoyer et de transmettre des paquets à des distances proches de l'utilisation finale du projet. Dans le cas où les résultats du test ne seraient pas satisfaisants, une nouvelle étude sur le matériel devrait être faite afin de trouver une solution fonctionnant dans les conditions d'utilisation du projet. Le test a été effectué le 13 Juillet 2018 et a permis de valider la phase \#1 avec succès ouvrant ainsi le développement pour la deuxième phase.

Pour plus d'information sur le test effectué et ses résultats voir le chapitre \ref{ch:test_1}.

\section{Phase \#2}

\begin{table}[htb]
\caption{Dates clefs phase \#2}
\label{tab:detail_phase_2}
\centering
\begin{tabular}{ l l l }
\toprule
Début & Validation \\
\midrule
14 Juillet 2018 & 7 Septembre 2018  \\
\bottomrule 
\end{tabular}
\end{table}

La deuxième phase du projet a été beaucoup plus conséquente en terme de travail à accomplir. Son objectif a été de valider la chaîne complète de communication du système, tous les éléments ont été développés dans leur forme simplifiée afin de pouvoir effectuer les tâches nécessaires au bon fonctionnement du système. Durant cette phase, l'accent a été mis sur la transmission de la position GPS, les autres paramètres ont été laissés de côté afin de pouvoir se concentrer sur les éléments les plus importants.

Pour le capteur, le système d'exploitation Zephyr a été pris en main afin d'y ajouter la configuration de la carte SODAQ One et les drivers manquants, $I^{2}C$ et GPS, ont été développés et testés afin de pouvoir interagir avec le modules GPS. Enfin le cœur de l'application du capteur a été écrit, qui consiste principalement à récupérer la position GPS au travers du bus $I^{2}C$ en utilisant le driver GPS et puis créer un paquet de données pour enfin l'envoyer grâce à la couche radio LoRa, ceci en boucle.

La structure de la base de données a été définie au moyen d'un diagramme UML et les scriptes associés permettant la création des différentes tables ont également été écrits. Le logiciel de gestion de base de données a ensuite été installé sur la passerelle et les tables créées.

Le serveur d'application a été développé d'avantage afin d'y intégrer la gestion du mode "race" qui permet la réception des données envoyées par le capteur, l'extraction des paramètres intéressants pour finalement aller écrire le tout dans la base de données.

L'application mobile a été mise en place, permettant la connexion à la base de données afin d'y récupérer les données nécessaires. La gestion de la carte bien connue de Google a été implémenté afin de pouvoir y afficher les positions récupérées depuis la base de données. Cependant, l'interface n'a été que très peu développée, l'accent étant mis sur le développement du fonctionnement de base plutôt que sur l'esthétique.

Cette phase a été terminée après un test en extérieur sur un anneau de distance similaire à une piste d'athlétisme permettant de valider le bon fonctionnement de la chaîne complète du système.

Pour plus d'information sur le test effectué et ses résultats voir le chapitre \ref{ch:test_2}.

\section{Phase \#3}

\begin{table}[htb]
\caption{Dates clefs phase \#3}
\label{tab:detail_phase_3}
\centering
\begin{tabular}{ l l l }
\toprule
Début & Validation \\
\midrule
20 Août 2018 & 22 Septembre 2018  \\
\bottomrule 
\end{tabular}
\end{table}

Une fois que le fonctionnement du système dans son ensemble a été validé, la troisième phase a pu commencer. Son objectif est la finalisation du système dans son entier, c'est à dire d'implémenter les éléments restants, comme l'acquisition du rythme cardiaque et de la cadence et leur écriture dans la base de données. L'interface graphique de l'application mobile ainsi que les fonctionnalités restantes comme la gestion des courses et la gestion graphique de la carte sont finalisées.

Il est à noter que cette phase a commencé plus tôt que la date du test formel de la phase \#2, ceci étant dû au fait que le test formel a dû être repoussé a plusieurs reprises à cause de problèmes techniques au niveau de la connexion de l'application mobile et de la passerelle. Afin de pouvoir respecter les contraintes de temps liées au projet, il a été décidé de commencer le développement relatif à la phase \#3 avant le passage formel du test de validation \#2.

Au terme de la phase \#3, un test faisant office de démonstration de l'utilisation du système a été fait afin de valider son fonctionnement global.

Pour plus d'information sur le test effectué et ses résultats voir le chapitre \ref{ch:test_3}.

\section{Environnement de développement}

Dans la mesure du possible, des outils modernes, performants, gratuits et open source ont été utilisés dans le cadre du développement du projet.  L'entier du travail effectué est hébergé sur la plateforme github.com, en accès libre, à l'adresse \rurl{github.com/leodido99/TravailBachelor}.

Les outils et librairies utilisés pour le développement des différents acteurs du projet figurent ci-dessous.

\begin{table}[htb]
\caption{Outils de développement généraux}
\label{tab:outils_dev_gen}
\centering
\begin{tabular}{p{3cm} p{5cm} p{6cm}}
\toprule
Nom & URL & Description \\
\midrule
LaTeX & \rurl{www.latex-project.org} & Système open source pour l'écriture de documents \\
git & \rurl{git-scm.com} & Outil open source de gestion de configuration \\
github & \rurl{www.github.com} & Plateforme d'hébergement gratuite pour des projets git \\
yEd & \rurl{www.yworks.com} & Outil de dessin de schéma \\
StarUML & \rurl{staruml.io} & Outil de création de diagramme UML \\
Fusion 360 & \rurl{www.autodesk.com} & Outil de dessin 3D \\
Matlab & www.mathworks.com & Outil d'analyse de données \\
\bottomrule 
\end{tabular}
\end{table}

\begin{table}[htb]
\caption{Outils de développement pour le capteur}
\label{tab:outils_dev_capteur}
\centering
\begin{tabular}{p{3cm} p{5cm} p{6cm}}
\toprule
Nom & URL & Description \\
\midrule
Eclipse & \rurl{www.eclipse.org} & IDE pour le développement en langage C/C++  \\
Zephyr Project & \rurl{www.zephyrproject.org} & Système d'exploitation temps réel sur lequel l'application du capteur se base \\
GNU ARM Embedded Toolchain & \rurl{developer.arm.com} & Toolchain fournissant gcc et les outils associés et permettant la compilation pour les cibles de types Arm Cortex-M \\
cmake & \rurl{cmake.org} & Évolution de make permettant la gestion du processus de compilation \\
BOSSA & \rurl{github.com/shumatech/BOSSA} & Outils permettant de programmer la mémoire des microcontrôleurs Atmel SAM \\
\bottomrule 
\end{tabular}
\end{table}

\begin{table}[htb]
\caption{Outils de développement pour la passerelle}
\label{tab:outils_dev_passerelle}
\centering
\begin{tabular}{p{3cm} p{5cm} p{6cm}}
\toprule
Nom & URL & Description \\
\midrule
Eclipse & \rurl{www.eclipse.org} & IDE pour le développement en langage C/C++  \\
GNU ARM RPI Toolchain & \rurl{github.com/raspberrypi} & Toolchain fournissant gcc et les outils associés et permettant la compilation pour les cibles de type ARMv7 comme le Raspberry Pi \todo{} \\
make & \rurl{www.gnu.org/software/make} & Permet la gestion du processus de compilation \\
Raspbian & \rurl{www.raspbian.org} & Distribution Linux basée sur Debian spécialement optimisée pour fonctionner sur le Raspberry Pi \\
PostgreSQL & \rurl{www.postgresql.org} & Système open source de gestion de base de données relationnelle  \\
PostGIS & postgis.net & PostGIS est une extension pour les bases de données PostgreSQL qui ajoute des fonctionnalités permettant de faire des requêtes et des opérations avec des positions géographiques \\
Single channel packet forwarder & \rurl{github.com/hallard/RPI-Lora-Gateway} & Logiciel packet forwarder permettant de récupérer les paquets depuis le module LoRa au travers du bus SPI \\
libpqxx & \rurl{pqxx.org/development/libpqxx} & API C++ pour client PostgreSQL permettant d'exécuter des requêtes sur une base de données \\
base64 & \rurl{github.com/Lora-net} & Classe permettant le décodage de données de type base64 \\
rapidjson & \rurl{rapidjson.org} & Librairie utilisée pour générer ou parser des chaînes de caractère de type json \\
\bottomrule 
\end{tabular}
\end{table}

\begin{table}[htb]
\caption{Outils de développement pour l'application mobile}
\label{tab:outils_dev_app}
\centering
\begin{tabular}{p{3cm} p{5cm} p{6cm}}
\toprule
Nom & URL & Description \\
\midrule
Android Studio & \rurl{developer.android.com/studio} & IDE pour le développement en langage java sur la plateforme Android  \\
Maps for Android SDK & \rurl{developers.google.com/maps} & SDK permettant la création et la manipulation de carte type Googlemaps  \\
Driver JDBC PostgreSQL & \rurl{jdbc.postgresql.org} & Driver JDBC pour PostgreSQL permettant l'exécution de requête sur des base de données de type PostgreSQL \\
\bottomrule 
\end{tabular}
\end{table}
