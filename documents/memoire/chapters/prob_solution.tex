
\chapter{Problèmes et solutions}\label{ch:prob_solutions}

Ce chapitre décrit les problèmes majeurs rencontrés pendant le développement du projet et les solutions qui ont été employées afin d'y pallier.

\section{Capteur: Driver $I^{2}C$ manquant}

Au début du développement du projet, j'ai décidé d'utiliser le système d'exploitation temps réel Zephyr. En consultant la page de la carte utilisant le même micro-contrôleur que le capteur, je n'ai malheureusement pas remarqué qu'il n'y avait aucune mentions d'un driver $I^{2}C$ existant. Je suis parti du principe qu'il en existait un. Lors du développement, afin de communiquer avec les modules GPS et accéléromètre, j'ai alors constaté ce manquement à mes dépends.

La solution a été simplement de m'armer de la data sheet du micro-contrôleur et d'en écrire un moi-même. Ce driver est décrit en détail dans le chapitre \ref{ch:drivers}.

\section{Capteur: Driver GPIO incomplet}

Lors de l'implémentation du module rythme cardiaque, il m'a fallu brancher une interruption sur un I/O afin de pouvoir détecter les battements du cœur. Après avoir écrit tout le code, je me suis rendu compte à l'exécution que le programme me retournait une erreur du type "not supported". En résumé, le driver GPIO SAM ne proposait pas cette fonctionnalité.

Afin de pallier à ce problème, j'ai implémenté cette fonctionnalité manquante. Pour ce faire, j'ai dû écrire un driver External Interrupt Controller qui permet de gérer le module du micro-contrôleur permettant d'ajouter des interruptions sur des lignes d'entrées/sorties. Une fois ce driver développé, il a fallu modifier le driver GPIO afin d’utiliser ce nouveau driver et ainsi permettre l'ajout d'interruption sur I/O.

\section{Passerelle: Problème de connexion réseau}

Pendant le développement du projet, un des domaines où j'ai rencontré beaucoup de problème est la configuration réseau de la passerelle. J'ai voulu utiliser l'interface WiFi du Raspberry Pi en conjonction avec un dongle WiFi afin d'avoir deux interfaces qui me permettent, sur une, de créer une access point afin de pouvoir me connecter avec mon téléphone mobile, et sur l'autre, de se connecter à un réseau WiFi. Je n'ai jamais réussi à faire fonctionner ce modèle comme je le voulais, probablement en partie dû à la mise à jour de Raspbian qui à rajouté pas mal de changements à ce niveau-là.

Après plusieurs dizaines d'essais de configuration différentes, j'ai décidé de ne plus utiliser le dongle et de le remplacer par un cable Ethernet, ce qui me permet de facilement me connecter à la passerelle depuis mon réseau personnel.

\section{Capteur: Interruption accéléromètre}

Afin de pouvoir détecter les pas du porteur du capteur, j'avais pris la décision d'utiliser un système proposé par l'accéléromètre placé sur la carte qui permet de le configurer afin qu'il déclenche une interruption lorsqu'une condition est détectée. Dans le cas du projet je voulais pouvoir déclencher une interruption lorsque l'accélération sur l'un des trois axes dépasse un certain seuil. Le driver LSM303AGR a été modifié afin de pouvoir permettre la configuration de ce type d'interruption, mais le problème est qu’aucune interruption n'était déclenchée lorsque le capteur était mis en mouvement. J’ai donc pris la décision de changer d'approche et d'utiliser un thread afin d'effectuer l'échantillonnage des données de l'accéléromètre.