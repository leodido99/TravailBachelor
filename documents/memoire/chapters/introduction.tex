\chapter{Introduction}

Ce travail de Bachelor vise à coupler deux disciplines qui m’intéressent particulièrement, d’une part les communications sans-fil dans le monde de l’informatique embarquée et de l’autre la course à pied. Étant moi-même coureur amateur et participant de temps en temps à des compétitions, je me suis demandé s’il serait possible de créer un système qui pourrait contribuer à impliquer davantage les spectateurs de ce genre de manifestation.

Le rythme auquel se déroule une compétition sportive, que ce soit de course à pied ou de cyclisme par exemple, associe la plupart du temps des moments de forte excitation avec des instants plus calmes. On peut découper ce genre d’événement en trois grosses phases. Avant le début de la compétition, il y a généralement beaucoup d’animation dans la zone de départ, les bénévoles s’occupant de l’événement mettent les dernières touches à l’organisation tout en remplissant les tâches requises par leur poste, les sportifs sont en train de s’échauffer et le public arrive peu à peu pour assister au départ de la compétition. Une fois que la course a commencé l’excitation retombe rapidement, les participants quittent la zone de départ pour rejoindre l’arrivée et les spectateurs vont souvent se déplacer le long du parcours pour les regarder passer ou au contraire se diriger directement vers la zone d’arrivée. Enfin, lorsque les sportifs arrivent petit à petit à la fin de la course, de plus en plus de gens se retrouvent dans l’aire d’arrivée, ce qui a pour effet de remettre l’ambiance.

Tout ceci fait que parfois le spectateur vit des moments de temps mort où il ne se passe pas forcément beaucoup de choses car il se trouve loin de l'action, de plus il est rare qu’il ait la possibilité de connaître la situation à un instant donné de la course. Seul les gros événements disposent parfois d’une retransmission télévisuelle, et à ma connaissance très peu proposent un outil utilisable par les spectateurs permettant de visualiser sur une carte la position des coureurs.

C’est sur cet axe que ce projet se positionne : proposer un outil interactif que les spectateurs d’événements sportifs peuvent utiliser à tout moment et de n’importe quel endroit pour connaître l’état actuel de la course.
