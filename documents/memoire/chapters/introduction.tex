\chapter{Introduction}

Le développement récent des technologies liées à l'Internet of Things permet la réalisation de systèmes embarquées intelligents, à faible coût et de petite taille. Aussi de plus en plus de carte électronique présentant des systèmes temps réels performant capable de communiquer avec une interface sans-fils existent sur le marché. L'envie de pouvoir développer un système sur cette base m'a amenée à porter une réflexion sur quel type de projet pourrait tirer partie d'une telle application. 

Étant moi même amateur de course à pied, je me suis posé la question de savoir si une application à base de capteur pourrait apporter une évolution dans ce sport ce qui me permettrait de combiner deux sujets qui m'intéressent particulièrement. Ceci m'a amené à l'idée poursuivie par mon travail de Bachelor, réaliser un système de suivi temps réel à base de la technologie LoRa afin d'être utilisé pendant des compétitions sportives. L'idée de base du projet est tirée du fait que pendant des compétitions, de course à pied mais cela est également applicable à d'autres sports comme le cyclisme par exemple, il n'est pas facile aux spectateurs d'avoir une vue d'ensemble de la situation de la course ou un classement précis ce qui peut rendre l'événement parfois ennuyeux ou difficile à suivre.

Afin d'essayer de rendre de tels événements plus vivant j'ai donc décidé de réaliser un système qui propose aux spectateurs l'utilisation d'une application mobile, avec laquelle il lui serait possible à tout moment de connaître la position actuelle des concurrents ainsi que d'autres informations intéressantes comme son temps de course, la distance parcouru ou son rythme cardiaque par exemple. 

Pour pouvoir proposer ces fonctionnalités, le système définit les éléments suivants. Un capteur porté par les sportifs qui est en charge de l'acquisition des différents paramètres et de leur transmission, une passerelle qui s'occupe de récupérer les données transmises par les capteurs, de les traiter et de les stocker dans une base de données et enfin de l'application mobile elle même qui permettra aux spectateurs de visionner une carte avec la position actuelle des compétiteurs ainsi que tous les paramètres acquis durant la course.

\todo{Add datasheet url to biblio}

\section{Énoncé du problème}

\todo{}