\chapter{Introduction}

Le développement récent des technologies liées à l'Internet of Things permet la réalisation de systèmes embarquées intelligents, à faible coût et de petite taille. Aussi de plus en plus de carte électronique présentant des systèmes temps réels performant capable de communiquer avec une interface sans-fils existent sur le marché. L'envie de pouvoir développer un système sur cette base m'a amenée à porter une réflexion sur quel type de projet pourrait tirer partie d'une telle application. 

Étant moi même amateur de course à pied, je me suis posé la question de savoir si une application à base de capteur pourrait apporter une évolution dans ce sport ce qui me permettrait de combiner deux sujets qui m'intéressent particulièrement. Ceci m'a amené à l'idée poursuivie par mon travail de Bachelor, réaliser un système de suivi temps réel à base de la technologie LoRa afin d'être utilisé pendant des compétitions sportives. L'idée de base du projet est tirée du fait que pendant des compétitions, de course à pied mais cela est également applicable à d'autres sports comme le cyclisme par exemple, il n'est pas facile aux spectateurs d'avoir une vue d'ensemble de la situation de la course ou un classement précis ce qui peut rendre l'événement parfois ennuyeux ou difficile à suivre.

Afin d'essayer de rendre de tels événements plus vivant j'ai donc décidé de réaliser un système qui propose aux spectateurs l'utilisation d'une application mobile, avec laquelle il lui serait possible à tout moment de connaître la position actuelle des concurrents ainsi que d'autres informations intéressantes comme son temps de course, la distance parcouru ou son rythme cardiaque par exemple. 

Pour pouvoir proposer ces fonctionnalités, le système définit les éléments suivants. Un capteur porté par les sportifs qui est en charge de l'acquisition des différents paramètres et de leur transmission, une passerelle qui s'occupe de récupérer les données transmises par les capteurs, de les traiter et de les stocker dans une base de données et enfin de l'application mobile elle même qui permettra aux spectateurs de visionner une carte avec la position actuelle des compétiteurs ainsi que tous les paramètres acquis durant la course.

\todo{Add datasheet url to biblio}

\section{Énoncé du problème}


Lors d’événements sportifs comme des courses à pied ou de vélo tout terrain, une fois le départ donné les spectateurs sont parfois loin de l’action pour une longue durée.

Afin de rendre ce temps mort plus intéressant, ce projet propose le développement d’un système de tracking des athlètes en direct. Grâce à un capteur placé sur chaque concurrent, il devient possible d’afficher sur une carte la situation globale de la course à tout moment.

L’objectif de ce système est de permettre de récupérer et centraliser la position GPS et le rythme cardiaque de chaque athlète équipé d’un capteur et d’afficher ces informations sur une carte géographique.

Le système est composé de 3 éléments distincts : un capteur, un gateway et une application.

Le capteur doit embarquer un capteur de rythme cardiaque et un système de positionnement GPS. De plus, il doit avoir sa propre source d’énergie avec une autonomie permettant son fonctionnement pendant l’entièreté d’une course.

Le gateway est le système qui récupère les données produites par les capteurs et les stocks dans une base de données située sur un serveur.

Les capteurs et le gateway communiqueront en utilisant le protocole LoRa (Long-Range) sur la bande de fréquence 868Mhz. De son côté le gateway se connecte à la base de données en utilisant un réseau WiFi.

L’application est en charge de l’affichage de la position des coureurs sur la carte ainsi que de leur rythme cardiaque. Une estimation de la vitesse et de la distance parcourue est également affichée. En plus l’application permet de rejouer une course qui s’est tenue dans le passée.

Une contrainte liée au capteur est qu’il doit être suffisamment petit pour ne pas gêner le sportif lors de son effort et être utilisable en extérieur.
