
% - Description de la passerelle
% 		- Le matériel
%		- Le packet forwarder
%		- Le serveur d'application
%		- La base de données

\chapter{Description de la passerelle}\label{ch:passerelle}


ADd info to install DB
% Install DB
% https://opensource.com/article/17/10/set-postgres-database-your-raspberry-pi
%sudo apt install postgresql libpq-dev postgresql-client 
%postgresql-client-common -y

%sudo su postgres

%createuser pi -P --interactive

%psql
%> create database test;

%psql test

%test=> create table people (name text, company text);

% Edit the PostgreSQL config file /etc/postgresql/9.6/main/postgresql.conf to uncomment the listen_addresses line and change its value from localhost to *. Save and exit.

% Edit the pg_hba config file /etc/postgresql/9.6/main/pg_hba.conf to change 127.0.0.1/32 to 0.0.0.0/0 for IPv4 and ::1/128 to ::/0 for IPv6. Save and exit.

% Install postgis
% sudo apt-get install postgis

% Install libpqxx
%sudo apt-get install libpqxx-dev

% Postgis Geometry vs Geography
% http://workshops.boundlessgeo.com/postgis-intro/geography.html

% SQL Requests
% Postgis geometry point
% Create point : ST_MakePoint(X,Y)
% Distance between two points: ST_Distance(point1, point2)
% Get latitude:  SELECT ST_X(column) from table;
% Get longitude: SELECT ST_Y(column) from table;

% Ajouter les requêtes SQL pour chaque éléments, à prendre des scriptes 


