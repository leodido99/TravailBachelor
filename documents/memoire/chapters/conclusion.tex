
\chapter{Conclusion}

Le travail effectué durant le développement du projet à montré que, d'un point de vue fonctionnel, LoRa permet tout à fait de remplir l'objectif fixé. La quantité de données que cette technologie permet de transférer à la fois est suffisante pour transmettre les informations requises pour le bon fonctionnement du système et les taux de transfert bas ne sont pas problématiques. En plus de cela il est très facile à mettre en place un petit système avec un émetteur et un récepteur à un coût très raisonnable qui permet de rapidement plonger dans le vif du sujet.

Même si LoRa est adéquat pour la situation, on notera que les limitations associés au taux de transfert, c'est à dire le respect du duty cycle imposé par la réglementation sur les bandes de fréquences libre, peut se révéler être un réel problème. En effet en prenant comme base une charge utile envoyée dans les paquets de 30 bytes, voir chapitre \ref{ch:paquet_donnee}, avec un facteur d'étalement maximum de sf12 si l'on veut maximiser la portée des données ce qui est notre cas, on peut calculer le temps qu'il faudra pour envoyer les données, le time on air, et donc déterminer le temps entre chaque envoie afin de respecter la limite. En utilisant la formule mis à disposition par la société Semtech on arrive à un temps de transfert de 1482.75 ms pour un seul paquet. Afin de respecter un duty cycle de 0.1\%, on ne pourra envoyer un paquet que toutes les 1482.75 s ou 24.71 min, pour une limite de 1\% on pourra envoyer un paquet toutes les 148.27 s c'est à dire toutes les 2.47 min, enfin pour un cycle fixé à 10\% c'est un paquet toutes les 14.8 secondes qu'il nous sera permis d'envoyer. \cite{lora_design_guide}

Actuellement le système utilise un interval de 15 s entre chaque message, cela est uniquement possible pour la bande de fréquence 869.4 à 869.65 Mhz, voir chapitre  \ref{ch:lora_lorawan} pour plus d'information à ce sujet. On constate que cette bande de fréquence est extrêmement étroite, on rappelle que afin de minimiser les interférences LoRa va changer de canal à chaque envoie, si l'on souhaite rester dans la bande mentionné cette technique est du coup diminuée. On peut donc dire que la réactivité du système est directement liée à la fréquence d'envoi des paquets puisque c'est seulement la réception de nouvelles données qui fait évoluer la position et les statistiques des coureurs, on comprend donc que le choix de ce paramètre est un élément très important dans la configuration du système et peut devenir un problème si l'on souhaite avoir une réactivité élevée dans l'application mobile.

Une autre problématique relative à LoRa est lié à la portée des messages envoyées. On peut souvent lire que la portée théorique des messages LoRa est d'environ 10 km, cela est vrai pour autant que les conditions optimales soit remplis. Comme étudié durant le test de la phase \#3 et le test de distance, voir chapitre \ref{ch:test_3} et \ref{ch:test_distance}, j'ai pu constater que si le signal est envoyé avec une puissance basse alors il est pratiquement impossible de communiquer si la vue entre le capteur et la passerelle est obstruée. En augmentant la puissance du signal les résultats sont beaucoup plus encourageant permettant une réception jusqu'au moins 1.2 km. Cet aspect mériterais d'être étudié d'avantage dans l'idée d'évolution de ce prototype, car les compétitions sportives sont rarement dans des régions dépourvue d'obstacle, on verra souvent ce genre d'événement en montagne ou en ville ou les obstacles sont légions. Même si ce problème est gênant, il existe néanmoins une solution pour diminuer les effets de ce problème, il est possible d'augmenter le nombre de passerelle utilisée par le système afin de pouvoir garantir une couverture optimale et d'assurer ainsi de minimiser le nombre de paquets perdus. 

Dans l'optique d'améliorer la qualité de la communication du capteur, l'utilisation de passerelles d'un réseau privé de concert avec des passerelles du système placé à des endroits clés peut être un réel atout. Un argument supplémentaires envers l'utilisation de la couche MAC LoRaWAN, qui est requises pour être capable de rejoindre des réseaux existant. D'avantage d'étude est nécessaire pour évaluer cette solution en plus de profondeur car ils existent des inconvénients, premièrement cette solution requiert des coûts supplémentaires car l'utilisation de ses réseaux est payante, il est également nécessaire de prendre garde aux quantités et taux de transferts qui sont souvent limités par les opérateurs.

L'utilisation du système d'exploitation temps réel Zephyr pour développer le firmware du capteur est bien adapté à l'application visée. De base Il propose toutes les mécaniques que l'on peut attendre d'un système de ce genre et propose une flexibilité très intéressante permettant de facilement changer ou modifier les configurations de l'électronique sur laquelle l'application est exécutée. Il est très facile de rajouter des drivers ou des sous-systèmes à sa guise afin de développer les éléments nécessaires pour son application. De plus la communauté qui est épaulée par de grandes structures comme Intel, NXP ou Linaro se rend disponible afin de répondre aux diverses questions que l'on pourrait avoir et la documentation disponible sur le site internet est très riche. Afin de contribuer moi même à l'évolution de Zephyr, j'ai comme projet futur de pouvoir ramener sur dépôt principal les développements qui ont été réalisés dans le cadre de ce travail de diplôme afin de pouvoir en faire profiter l'entier de la communauté.

La plateforme Android est très puissante et propose des centaines de fonctionnalités et d'énormes quantités de documentations ce qui permet de créer des application mobile moderne et performante. En particulier le simulateur de la plateforme qui est fournis avec le logiciel Android Studio, qui a été utilisée pendant le développement du projet, est un outil indispensable afin de faciliter la mise au point de son programme. Il est également très facile d'exécuter ensuite son application sur n'importe quel téléphone muni de ce système d'exploitation pour s'assurer du fonctionnement en conditions réelles.

Pour ce qui concerne l'évolution futur du système, deux aspects devraient être approfondis en priorités. Le premier est de faire d'avantage de tests, de toute sorte, afin de pouvoir mieux caractériser les performances du système et les effets exactes des paramètres qu'il est possible d'altérer, en particulier en ce qui concerne la communication LoRa. Deuxièmement la modernisation de l'interface entre la base de données et l'application mobile semble être un point qui mérite des améliorations. Il est nécessaire d'utiliser les techniques de l'état de l'art dans le domaine et ainsi assurer que le système soit capable de soutenir une charge accrue amener par d'avantages d'utilisateur du système et un nombre élevé de capteurs. Enfin l'algorithme du calcul de la cadence doit être finalisé, en effet,  le manque de temps ne m'a malheureusement pas permis de pouvoir développer cette partie autant que je l'aurais voulu.

Le développement de ce travail m'a permis d'apprendre beaucoup de chose sur différents domaines. J'ai pu explorer l'utilisation de la technologie LoRa pour transmettre des données et j'ai eu le plaisir de pouvoir toucher à plusieurs langages de programmation allant du bas niveau, avec le C, jusqu'au haut niveau avec java et C++. Il m'a été possible de développer une base de données et de faire la connaissance du développement pour la plateforme Android. La mise en route du capteur et l'écriture de plusieurs drivers m'ont permis de découvrir le cœur du système d'exploitation temps réel Zephyr et également les microcontrôleurs ARM. Enfin je souhaiterais remercier chaleureusement ma compagne, qui a fait office de cobaye à mainte reprise et sans qui ce projet n'aurait pas pu être possible.