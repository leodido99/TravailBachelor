\chapter{Dossier de gestion}

Le dossier de gestion regroupe les activités effectuées pendant toute la durée du développement du projet.

\begin{multicols}{2}

\gestiondate{01.07.2018}
Installation du Raspberry Pi et du Dragino HAT
Installation Arduino IDE
Récupération du code du packet forwarder

\gestiondate{02.07.2018}
Modification fichier de configuration
Exécution packet forwarder
Setup Eclipse C++
Début codage serveur d'application basique

\gestiondate{03.07.2018}
Design diagramme de class serveur d'application

\gestiondate{04.07.2018}
Avancement serveur d'application

\gestiondate{05.07.2018}
Avancement serveur d'application (parsage paquet UDP)

\gestiondate{06.07.2018}
Installation d'Eclipse pour le Raspberry Pi avec XWindow
Ne fonctionne pas très bien, abandon de l'idée
Utilisation de scp pour copier les fichiers
Installation de Ubuntu sur ordinateur de développement

\gestiondate{07.07.2018}
Script connection Raspberry pi
Refactor class lora\_udp\_pkt
Débuggage des nouvelles classes
Implémentation décodage donnée base64
Premier paquet reçu

\gestiondate{08.07.2018}
Commentaires dans le code
Création classe Shell
Début codage avec Arduino IDE pour le capteur

\gestiondate{09.07.2018}
Début mise en place du mémoire
Rédaction de l’explication liée au test 1 (capteur + passerelle)

\gestiondate{10.07.2018}
Création de la classe test\_mode\_record pour sauver les données enregistré par le test\_mode
Codage de la classe vector\_reader permettant de retirer un certains nombre de byte d’un vecteur
Création de la classe shell\_command qui permet la définition de commande et de leurs comportement
Implémentation / debug de la première commande (test)

\gestiondate{11.07.2018}
Configuration du Raspberry Pi avec deux interfaces WiFi une pour client internet l’autre pour AP
Test RPi en alim depuis le mac -> OK
Test battery SODAQ -> Mauvais connecteur -> Recommander un autre
Mise en place zephyr sur Ubuntu

\gestiondate{12.07.2018}
Travail sur la configuration de la carte SODAQ One v3 sur Zephyr
Soudage de pin sur le capteur pour accéder aux pins UART
Installation câble FTDI
Test LED OK
Test UART OK

\gestiondate{13.07.2018}
Dév de la board SODAQ One sur Zephyr.
Essai d’interfacage avec le chip LoRa, UART ne fonctionne pas reste en boucle d’interruption.
Début écriture du module RN2483\_lora
Débug RN2483 LoRa. envoie commande et récup status = OK

\gestiondate{14.07.2018}
Codage LoRa radio tx
Finir la configuration de la carte (GPS + Magnetometer)
Investigation pour écriture du driver I2C
Mise en place environnement pour dev Zephyr (Fork)

\gestiondate{15.07.2018}
Modifications des fichiers nécessaire pour l’ajout du driver I2C à Zephyr
Début codage du driver I2C

\gestiondate{16.07.2018}
Travail sur I2C. Codage driver
Début écriture driver LSM303AGR

\gestiondate{17.07.2018}
Travail sur I2C. 
Lecture note pour I2C: Atmel-42631-SAM-D21-SERCOM-I2C-Configura.pdf

\gestiondate{18.07.2018}
Première version qui fonctionne du I2C
Codage driver LSM303AGR pour voir si cela fonctionne bien
Ajout de la configuration pour SODAQ One v3 au fork Zephyr
Modification des configurations projet pour utiliser la board SODAQ One v3
Débug codage driver UBlox EVA8M

\gestiondate{19.07.2018}
Continue codage driver UBlox EVA8M
Driver GPS fonctionne correctement
Continue codage driver LSM303AGR

\gestiondate{20.07.2018}
Test avec battery

\gestiondate{21.07.2018}
Test capteur/gateway à Planeyse. 
5 Tests différents: SF7/SF12 - Marche le long de la piste, SF7/SF12/SF7-POWER10 distance (marche en ligne droite environ 200m et retour)

\gestiondate{22.07.2018}
Écriture du chapitre test phase \#1
Début écriture introduction
Mise en place chapitres du mémoire
Début écriture chapitre capteur

\gestiondate{23.07.2018}
Réécriture du driver RN2483 LoRa + test

\gestiondate{24.07.2018}
Modification de la passerelle pour décoder les paquets “course”
Amélioration shell passerelle

\gestiondate{25.07.2018}
Envoie des paquets du capteur en big endian
Correction de l’affichage des paquet sur la passerelle
Installation postgresql sur la passerelle

\gestiondate{26.07.2018}
Mise en place database depuis StarUML
Installation PostGis et tests des différents types et fonctions
Création base de données race\_tracker\_db avec tables
Création des requêtes pour effectuer les diverses opérations (ajout compétiteur, ajout compétition etc..)

\gestiondate{27.07.2018}
Ecriture du script pour transformer un fichier gpx en requêtes SQL

\gestiondate{28.07.2018}
Codage serveur d'application "race mode"

\gestiondate{29.07.2018}
Ecriture rapport (description capteur)

\gestiondate{30.07.2018}
Installation et test lib pqxx
Début codage class interface DB

\gestiondate{31.07.2018}
Codage class interface DB
Modification de la DB (ajout active Competition)
Ecriture requête insertion data point dans DB

\gestiondate{01.08.2018}
Ecriture + test ecriture des positions GPS dans la database depuis gateway
Ajout nouvelle fonction race sensor shell pour changer l’intervalle de temps entre deux messages

\gestiondate{02.08.2018}
Rédaction du résumé du travail de Bachelor
Rédaction chapitre passerelle du mémoire

\gestiondate{03.08.2018}
Installation et configuration Android studio
Installation et configuration KVM Virtual Machine
Test lancement emulateur

\gestiondate{04.08.2018}
Installation driver JDBC
Configuration Android studio pour driver JDBC Postgresql
Ecriture de classe interface DB RaceTrackerDb

\gestiondate{05.08.2018}
Ecriture RaceVieverSelector (Lecture des compétitions depuis la DB, affichage dans UI, séléction course -> Changement d’activité)
Affichage de la carte et markage de la position de la course

\gestiondate{06.08.2018}
Ecriture requête last data\_point
Application mobile: RaceViewer
Retravail de la classe RaceTrackerDB au niveau de l’exécution des requêtes et la récupération des résultats

\gestiondate{07.08.2018}
Debug exécution de plusieurs requête
Ecriture RaceTrackerDBAsyncTask
Gestion exception sur requête SQL

\gestiondate{08.08.2018}
Gestion affichage de la position depuis la DB dans l’application mobile

\gestiondate{10.08.2018}
Description classes du serveur d’application dans le mémoire
Test avec capteur + passerelle + appli mobile
Première version complète du système fonctionnelle avec uniquement position GPS

\gestiondate{11.08.2018}
Re-écriture de la gestion et l’affichage des positions
Test nouvelle version avec points GPS obtenu depuis fichier GPX

\gestiondate{15.08.2018}
Lecture data sheet module rythme cardiaque
Soudage cable sur le module
Soudage rangée de pins sur le SODAQ One

\gestiondate{16.08.2018}
Correction de la configuration Wifi de la passerelle (wlan0 = connection normal sur réseau local, wlan1 = Access point)

\gestiondate{17.08.2018}
Correction du résumé du TB et enregistrement sur le site
Mail M. Bressy pour relecture et prise rendez-vous
Ecriture mémoire chapitre Passerelle

\gestiondate{18.08.2018}
Codage module Rythme Cardiaque sur le SODAQ One
Debug Hot spot WIFI
Driver GPIO Zephyr ne peut pas utiliser les interruptions (Non implémenté)

\gestiondate{19.08.2018}
Ecriture mémoire, partie Passerelle et Base de données

\gestiondate{20.08.2018}
Investigation SEGGER J-Link EDU debugger
Connection debugger -> Ne fonctionne pas (VTRef = 0V)
Codage GPIO interrupt pour SAMD21

\gestiondate{22.08.2018}
Ecriture driver external interrupt controller Zephyr
Modification driver GPIO pour utiliser external interrupt controller

\gestiondate{23.08.2018}
Debug driver external interrupt controller

\gestiondate{24.08.2018}
Debug problem AP WiFi ne fonctionne pas

\gestiondate{25.08.2018}
Update RPi firmware -> AP WiFi fonctionne
Ecriture script start-up pour la passerelle
Debug driver External Interrupt Controller

\gestiondate{27.08.2018}
Interrupt fonctionne ! Réception des premiers battements
Fix bug driver GPIO list callback

\gestiondate{28.08.2018}
Fin écriture module rythme cardiaque et test
Mise en place module cadence
Nettoyage divers modules

\gestiondate{29.08.2018}
Création affiche
Ecriture mémoire, chapitre développement

\gestiondate{30.08.2018}
Ajout timestamp au paquet de donnée
Test timestamp et transmission rythme cardiaque à la passerelle puis enregistrement dans la base de données -> SUCCESS!

\gestiondate{01.09.2018}
Backup SD Card RPi

\gestiondate{02.09.2018}
Essaie de test en extérieur -> Échec pas de connection à la passerelle…

\gestiondate{03.09.2018}
Reconfiguration de la passerelle pour avoir un AP fonctionnel

\gestiondate{04.09.2018}
Essaie de test en extérieur -> Echec le réseau de l’AP n’apparaît pas. Après plusieurs essais, avec un hotspot créé depuis un téléphone mobile avec le même nom de réseau que le réseau wlan0, le réseau de l’AP est brièvement apparue mais que pour un cours instant. Problème de configuration au niveau de la passerelle à revoir…

\gestiondate{05.09.2018}
Investigation problème AP

\gestiondate{06.09.2018}
Décision de laisser tomber les deux wifi en parallèle. Simplement utiliser le wifi du RPI pour créer l’AP et si besoin de connection pour debug utiliser le cable ethernet
Configure du RPI, test avec et sans cable ethernet, tout fonctionne bien.
Développement interface graphique application mobile

\gestiondate{07.09.2018}
Retouche affiche pour finalisation
Ecriture mémoire chapitre Application mobile, driver External Interrupt Controller.

\gestiondate{08.09.2018}
Développement interface graphique Android

\gestiondate{09.09.2018}
Refactor interface DB app Android
Encapsulation des requête pour mieux contrôler la fermeture des connections
Ajout rythme cardiaque et cadence sur interface

\gestiondate{10.09.2018}
Fini refonte interface DB + test
Nettoyage + commentaires

\gestiondate{11.09.2018}
Create fonction de gestion dans l’application mobile
Finalisation activité création de course avec écriture dans DB
Activité création de compétiteur et écriture dans la DB

\gestiondate{12.09.2018}
Finalisation insertion participant dans DB
Refactor ViewRaceSelector pour retourner une course sélectionné
Création nouvelles activités pour inscriptions: RegistrationsActivity et RegistrationEditActivity
Quelques tests des différentes fonctions

\gestiondate{13.09.2018}
Travail sur interface graphique Android partie gestion de course
Gestion début/fin de course

\gestiondate{14.09.2018}
8h00: Rendez-vous avec le conseiller pour signature affiche plus discussion générale
Ecriture mémoire, chapitre application mobile, documentation des classes
Finitions chapitre capteur

\gestiondate{15.09.2018}
Développement du mode Replay
Test mode replay avec donnée Fyne Tera -> Fonctionne bien
Test accéléromètre sur le capteur, correction de quelques problèmes
Récupération de données accéléromètre en marchant avec le capteur
Configuration interruption sur dépassement de seuil sur un certains axe
Test pas concluant donc replis sur solution sampling en polling

\gestiondate{16.09.2018}
Encore quelques test cadence interrupt, ne marche toujours pas
Ajout temps minimum entre deux détection de pas
Intégration de cadence test dans code capteur
Debug problème accès concurrent sur i2c -> fixed
Test envoie de paquet depuis le capteur pendant 2h30 -> OK
Clean-up debug application mobile
Test de quelques cas d’erreur sur gateway
Clean-up fix quelques bug sur gateway
Début design boîte sur Fusion 360

\gestiondate{17.09.2018}
Avancé design boîte. Design fermeture de la boîte
Réajustement de la taille de la boîte (erreur de taille du SODAQ)
Ajustement de la position du switch on/off
Ajustement position antenne GPS

\gestiondate{18.09.2018}
Design boîte. Premier test impression, petit erreur de dimension sur le modèle de la carte SODAQ.
Corrections apportées suite aux problèmes
Ajout de support pour la carte
Correction et finalisation de la partie pour l’attache élastique

\gestiondate{19.09.2018}
Fin design boîte. Test dimensions pour interrupteur, batterie etc..

\gestiondate{20.09.2018}
Ecriture mémoire:
Chapitre design boîte capteur
Chapitre description LoRa \& LoRaWAN

\gestiondate{21.09.2018}
Ecriture mémoire:
Finaliser LoRa \& LoRaWAN
Nettoyage bibliographie
Chapitre environnement de développement
Chapitre cadence analyse
Chapitre application mobile, accès à la base de données
Chapitre problèmes \& solutions
Début chapitre test phase 2
Test avec switch pour éteindre le capteur. Après plusieurs essaies impossible de réussir à éteindre le capteur.
Soudage cable switch
Soudage module rythme cardiaque avec fil souple
Test assemblage de la boîte

\gestiondate{22.09.2018}
Finalisation de l’assemblage du capteur
Création de la bande élastique permettant de fixer le capteur au bras
Réparation de la soudure du module rythme cardiaque et isolation avec scotch
Nettoyage des messages de debug du firmware du capteur et version finale
Fix un bug dans l’application mobile lors de la création d’une course
Tag de la version 1.0 sur git!
Test phase \#3 sur piste d’athlétisme -> Succès !
Test de communication LoRa avec obstacle
Ecriture mémoire chapitre test phase \#2

\gestiondate{23.09.2018}
Fin écriture mémoire chapitre test phase \#2
Ecriture chapitre test phase \#3
Ecriture chapitre évolution du système
Début écriture de la conclusion
Recherche d’un endroit pour test de distance

\gestiondate{24.09.2018}
Test de distance vers l’aérodrome de Colombier -> Peut concluant perte de la trace relativement rapidement. Deuxième essai demain avec augmentation de la puissance du signal.
Ecriture chapitre test de performance

\gestiondate{25.09.2018}
Fix bug sélection de l’emplacement de la course pendant la création de course
Test de distance vers l’aérodrome de Colombier, édition numéro 2.
Fin écriture chapitre test de performance
Ecriture de la conclusion

\gestiondate{26.09.2018}
Fin écriture de la conclusion
Photo capteur et passerelle
Modification introduction
Finalisation de l'écriture du rapport

\end{multicols}